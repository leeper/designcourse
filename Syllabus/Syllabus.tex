\documentclass[12pt,a4paper]{article}
\usepackage[margins=1in]{geometry}
\usepackage{natbib}

\usepackage{setspace,mdwlist,comment}
\setlength{\marginparwidth}{.5in}
\setlength{\parindent}{0in}

% mini table of contents
\usepackage{minitoc}
\dosecttoc % make section toc
\setcounter{secttocdepth}{2} % subsection depth
\renewcommand{\stctitle}{} % no title
\nostcpagenumbers

\usepackage{natbib}
\usepackage{bibentry}
\newcommand{\reading}[2][]{\noindent -- {#1}\bibentry{#2}.\vspace{.25em}\\}
\newcommand{\textbook}[2][]{\noindent -- {#1}#2.\vspace{.25em}\\}
\newcommand{\thomas}{\vspace{1em}\noindent Instructor: Thomas\\}
\newcommand{\david}{\vspace{1em}\noindent Instructor: David\\}
\newcommand{\seealso}{\noindent \emph{See Also:}}
\newcommand{\topic}[1]{\noindent \textbf{#1}\\}

\usepackage{hyperref}
\hypersetup{
    bookmarks=true,         % show bookmarks bar?
    unicode=false,          % non-Latin characters in Acrobat’s bookmarks
    pdftoolbar=true,        % show Acrobat’s toolbar?
    pdfmenubar=true,        % show Acrobat’s menu?
    pdffitwindow=false,     % window fit to page when opened
    pdfstartview={FitH},    % fits the width of the page to the window
    pdftitle={Syllabus: Research Design in Political Science},    % title
    pdfauthor={Thomas J. Leeper},     % author
    pdfsubject={Government} {Political Science},   % subject of the document
    pdfnewwindow=true,      % links in new window
    pdfborder={0 0 0}
}

\title{Syllabus for \textit{Research Design in Political Science}\\Department of Government\\LSE\\2015-2016}

\begin{document}
\nobibliography*

\maketitle

\faketableofcontents

\begin{minipage}[b]{0.5\linewidth}
Instructor\\
Thomas J. Leeper\\
Department of Government\\
Connaught House 3.21\\
\href{mailto:thosjleeper@gmail.com}{thosjleeper@gmail.com}\\
\end{minipage}
\begin{minipage}[b]{0.5\linewidth}
GTA\\
TBD\\
\end{minipage}


%\section{Introduction}
The course will introduce students to the fundamentals of research design in political science. The course will cover a range of topics, starting from the formulation of research topics and research questions, the development of theory and empirically testable hypotheses, the design of data collection activities, and basic qualitative and quantitative data analysis techniques. The course will address a variety of approaches to empirical political science research including experimental and quasi-experimental designs, large-n survey research, small-n case selection, and comparative/historical comparisons. As a result, topics covered in the course will be varied and span all areas of political science including political behaviour, institutions, comparative politics, international relations, and public administration.

Every week will involve a lecture followed by a class with a Graduate Teaching Assistant. 

\clearpage
\section{Objectives}
The learning objectives for the course are as follows. By the end of the course, students will be able to:

\begin{enumerate*}
\item Identify interesting political science research questions and formulate theories and hypotheses that answer them
\item Describe and operationalize concepts from political science theories
\item Evaluate the strengths and weaknesses of different approaches to empirical research
\item Apply political science theories to the design of original research
\end{enumerate*}

\section{Learning Assessment and Feedback}

Students will be evaluated through (1) a 2-hour written exam covering the full breadth of course content and (2) a 3000-word written paper applying course material in the form of a research design proposal. The final mark will reflect an equal weighting of both forms of assessment.

The written exam covers the full breadth of material from the course and will test students' knowledge of course content, including concept definition, the appraisal of political science theories, the generation of hypotheses, and --- most importantly --- the appropriateness of different research designs for answering specific research questions. This will count for 50\% of the final mark.

The individual research design paper should outline the basic elements of a novel research project, namely a research question, theoretical contribution, testable hypotheses, and a description of the proposed data collection and analysis. Unlike the written exam, this paper should focus narrowly on a topic of the student's choice and display a greater depth of understanding of a smaller set of ideas raised in the course. This will count for 50\% of the final mark.

As formative work in preparation for both exam forms, students will complete short ``problem set'' assignments, approximately every other week (see course schedule for details), which allow them to apply material from the course to concrete political science examples (e.g., identifying design elements of a published research paper; proposing strategies for answering a given research question, etc.). While these formative assessments do not count toward the final mark, they provide an opportunity for peer and instructor feedback.


\clearpage
\section{Reading Material}

\section{Course Website}




\clearpage
\section{Schedule}
The general schedule for the course is as follows. Details on topics covered and the readings for each week are provided on the following pages.

\secttoc

\clearpage

% Introduction


Mahoney, J. & Goertz, G. 2006. "A Tale of Two Cultures: Contrasting Quantitative and Qualitative Research." Political Analysis 14: 227-249.

% Concepts
Goertz

Gerring

% Measurement and measurement validity
Adcock, R. & Collier, D. 2001. “Measurement Validity: A Shared Standard for Qualitative and Quantitative Research.” American Political Science Review 95: 529-546.

% Theory testing (hypothesis generation and testing; philosophy of science)
% ??? positivism
% ??? publication bias?
% ??? 


% Description
Athaus, Scott L., and Devon M. Largio, “When Osama Became Saddam: Origins and Consequences of the Change in America’s Public Enemy #1,” 

% Visualization


% Case studies

% External validity; Inference from sample to population
% survey sampling and representativeness
papers on online panels?
% replication


% Causality
% Process-Tracing
Collier



% Comparative-historical methods
James Mahoney, “Nominal, Ordinal, and Narrative Appraisal in Macrocausal Analysis,” American Journal of Sociology 104:4 (January 1999), pp. 1154-1169. 

Doner, Richard F., Bryan K. Ritchie, and Dan Slater, “Systemic Vulnerability and the Origins of Developmental States: Northeast and Southeast Asia in Comparative Perspective,” International Organization 59 (Spring 2005), pp. 327-361.

Geddes, Barbara. 1990. “How the Cases You Choose Affect the Answers the Answers You Get: Selection Bias in Comparative Politics.” Political Analysis 2: 131-150.

% Large-n Observational methods (matching)
Morgan and Winship?

% OVB/conditioning --> Experiments
Gerber, A. S. & Green, D. P. 2008. “Field Experiments and Natural Experiments.” In Box-Steffensmeier, J. M.; Brady, H. E. & Collier, D. (Eds.), Oxford Handbook of Political Methodology, Oxford University Press.

% Quasi-Experiments (ITS/DID/RDD)
Connecticut Crackdown on Speeding

% Correlation and regression for causal inference

% Repeated cross-sections and panel designs 



% Theory development
% Explanation versus prediction
1. Shmueli, Galit. 2010. “To Explain or to Predict?” Statistical Science 25(3): 289-310.
2. Stevens, Jacqueline. 2012. “Political Scientists are Lousy Forecasters.” New York Times, Sunday June 26th, 2012.
URL: http://goo.gl/Iiq03L.
3. Dickinson, Matthew. 2012. “No, Political Scientists are NOT Lousy Forecasters – In fact, they are Pretty Good.”
Presidential Power: A Nonpartisan Analysis of Presidential Politics, June 26th, 2012. URL: http://goo.gl/8O8j3N.
4. Henry Farrell. June 24, 2012. “Why the Stevens Op-Ed is Wrong.” The Monkey Cage. URL:
http://themonkeycage.org/blog/2012/06/24/why-the-stevens-op-ed-is-wrong/. 

% Developing theories
Gerring

% Formal theory (and EITM)
% Deducing hypotheses from other kinds of theories


% Problem Sets
% 1. Concepts
% 2.
% 3.
% 4. 
% 5.
% 6.
% 7.
% 8.
% 9.
% 10.




\subsection{Introduction}
\emph{}

\thomas

\subsection*{Lecture}

\begin{itemize*}
\item 
\end{itemize*}

\subsubsection*{Readings}

\seealso




% load bibtext, but don't generate a bibliography
\bibliographystyle{plain}
\nobibliography{Syllabus}


\end{document}
