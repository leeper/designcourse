\documentclass[14pt]{beamer} %Makes presentation
%\documentclass[handout]{beamer} %Makes Handouts
\usetheme{Singapore} %Gray with fade at top
\useoutertheme[subsection=false]{miniframes} %Supppress subsection in header
\useinnertheme{rectangles} %Itemize/Enumerate boxes
\usecolortheme{seagull} %Color theme
\usecolortheme{rose} %Inner color theme

\definecolor{light-gray}{gray}{0.75}
\definecolor{dark-gray}{gray}{0.55}
\setbeamercolor{item}{fg=light-gray}
\setbeamercolor{enumerate item}{fg=dark-gray}

\setbeamertemplate{navigation symbols}{}
%\setbeamertemplate{mini frames}[default]
\setbeamercovered{dynamics}
\setbeamerfont*{title}{size=\Large,series=\bfseries}

%\setbeameroption{notes on second screen} %Dual-Screen Notes
%\setbeameroption{show only notes} %Notes Output

\setbeamertemplate{frametitle}{\vspace{.5em}\bfseries\insertframetitle}
\newcommand{\heading}[1]{\noindent \textbf{#1}\\ \vspace{1em}}

\usepackage{bbding,color,multirow,times,ccaption,tabularx,graphicx,verbatim,booktabs,fixltx2e}
\usepackage{colortbl} %Table overlays
\usepackage[english]{babel}
\usepackage[latin1]{inputenc}
\usepackage[T1]{fontenc}
\usepackage{lmodern}

%\author[]{Thomas J. Leeper}
\institute[]{
  \inst{}%
  Department of Government\\London School of Economics and Political Science
}

\usepackage{tikz}
\usetikzlibrary{shapes,arrows}

\title{Concepts: ``I'll know it when I see it''}

% Before we can study something we need to know what that ``something'' is. This is concept definition. How do we define concepts and how do we separate different concepts from one another?

\date[]{}

\begin{document}

\frame{\titlepage}

\frame{\tableofcontents}

\section{Quick Review}

\frame{\tableofcontents}
\frame{\tableofcontents[currentsection]}


\frame{
\frametitle{Fundamental problem of causal inference}

\Large We can only observe any given unit in one reality!

}


\frame{
\frametitle{In Political Science}

\begin{itemize}\itemsep0.5em
\item Causal inference is about searching for appropriate counterfactuals

\item \textit{Causal effect}: Difference in an outcome variable between two counterfactuals

\item \textit{Causal inference}: A belief that an event or variable exerts a causal effect on an outcome
\end{itemize}

}


\section{Concepts}
\frame{\tableofcontents[currentsection]}

\frame{
\frametitle{Concepts}

\begin{itemize}
\item Definition: The words and ideas that we use to describe the world

\vspace{1em}
\item Why do we care? We cannot theorize or study a phenomenon until we know what it is
\end{itemize}
}


\frame{
\frametitle{Ogden and Richard's Triangle}

\begin{tikzpicture}
% meaning (set of ideas we associate with a concept; the essential constitutive elements)

% label (word used to designate a concept)

% cases (entities that are instances or cases of a concept)

\end{tikzpicture}
}


% distinguish concept definition from causal theorizing



\frame{

% ambiguity
% % label connected to multiple meanings
% % meaning connected to multiple labels

% vagueness
% % lack of a good definition


}

\frame{
\frametitle{Approaches to Concepts}

% Classical Approach

% Family Resemblence

% Gerring's criteria
}



\subsection{Classical Approach}
\frame{\tableofcontents[currentsubsection]}


\frame{
\frametitle{}
}


% Gerring distinguishes between minimal (classical approach); maximal (ideal types); cumulative (more or less of a concept)
% the last of these will be particularly helpful when we start thinking about operationalization or measurement next week


\subsection{Family Resemblence}
\frame{\tableofcontents[currentsubsection]}

\frame{
\frametitle{Family Resemblence}

\begin{itemize}
\item Classical approach focuses on \textit{necessary} elements
\item Some concepts have no necessary elements but are still meaningful % Game
\item We might also think about elements that are \textit{sufficient} to establish membership
\end{itemize}
}

\frame{
\frametitle{Boolean Logic}

\begin{itemize}
\item \textbf{AND}: necessary
\item \textbf{OR}: sufficient
\end{itemize}

}

% examples: game, democracy


\subsection{Gerring's Criteria}
\frame{\tableofcontents[currentsubsection]}


\frame{
\frametitle{Gerring's Criteria}

\begin{enumerate}
\item Resonance
\item Domain/scope % ladder of generality
\item Consistency
\item Fecundity
\item Differentiation
\item Causal utility
\item Operationalization (more on this next week)
\end{enumerate}

}

% divide these up and have someone present each one



\frame{
\frametitle{}
}



\appendix
\frame{}

\end{document}
