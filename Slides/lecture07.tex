\documentclass[14pt]{beamer} %Makes presentation
%\documentclass[handout]{beamer} %Makes Handouts
\usetheme{Singapore} %Gray with fade at top
\useoutertheme[subsection=false]{miniframes} %Supppress subsection in header
\useinnertheme{rectangles} %Itemize/Enumerate boxes
\usecolortheme{seagull} %Color theme
\usecolortheme{rose} %Inner color theme

\definecolor{light-gray}{gray}{0.75}
\definecolor{dark-gray}{gray}{0.55}
\setbeamercolor{item}{fg=light-gray}
\setbeamercolor{enumerate item}{fg=dark-gray}

\setbeamertemplate{navigation symbols}{}
%\setbeamertemplate{mini frames}[default]
\setbeamercovered{dynamics}
\setbeamerfont*{title}{size=\Large,series=\bfseries}

%\setbeameroption{notes on second screen} %Dual-Screen Notes
%\setbeameroption{show only notes} %Notes Output

\setbeamertemplate{frametitle}{\vspace{.5em}\bfseries\insertframetitle}
\newcommand{\heading}[1]{\noindent \textbf{#1}\\ \vspace{1em}}

\usepackage{bbding,color,multirow,times,ccaption,tabularx,graphicx,verbatim,booktabs,fixltx2e}
\usepackage{colortbl} %Table overlays
\usepackage[english]{babel}
\usepackage[latin1]{inputenc}
\usepackage[T1]{fontenc}
\usepackage{lmodern}

%\author[]{Thomas J. Leeper}
\institute[]{
  \inst{}%
  Department of Government\\London School of Economics and Political Science
}

\usepackage{tikz}
\usetikzlibrary{shapes,arrows}

\title{Case Studies}

% Case studies are in-depth examinations of a single manifestation of a political phenomenon and are one of the most common methods of inquiry in political science. What can we do with case studies? How do they help us to understand politics?


\date[]{}

\begin{document}

\frame{\titlepage}

\frame{\tableofcontents}



\section[Hypothesis Testing]{Conclude Hypothesis Testing}
\frame{\tableofcontents[currentsection]}



\frame{

\frametitle{Fearon's Counterfactuals}

\begin{itemize}\itemsep0.5em
\item Sometimes we cannot test our hypothesis with actual observations % reasons for this: confounding, no counterfactual case, multiple causality
\item What does Fearon suggest we do?
\end{itemize}


}

% thinking counterfactually
	% can we observe the counterfactual (in another case, in the same case at another time, in a collection of other cases)?
	% if not, can we think about what that counterfactual might look like? 
	% see Fearon's use of ``counterfactual method''; thought experiments; hypothetical reasoning



\frame{

\frametitle{A Good Test}

\small

\begin{itemize}\itemsep-0.1em
\item Correct level of analysis
\item Within scope conditions of theory
\item Well-defined concepts
\item Measures of high construct validity, accuracy, and precision
\item Possible to observe any correlation between potential cause and outcome
\item Consistent with or an improvement upon past methods
\item Test using different data than data used to generate theory % distinguish between training set and test set
\end{itemize}

}


\frame{

\frametitle{Some Testing Challenges}

\begin{enumerate}\itemsep0.5em
\item Deterministic and probabilistic causality
\item Effect heterogeneity
\item Multiple causation
\item Equifinality
\item Confirmation or disconfirmation bias
\end{enumerate}

}


\frame{}



\section[PS1]{Problem Set 1}
\frame{\tableofcontents[currentsection]}


\frame{

\frametitle{Small Points}

\begin{enumerate}\itemsep0.5em
\item Follow the formatting guidelines
\item Do not copy the questions into the assignment
\item Be succinct but expressive and clear
\item Be anonymous (filename and contents)
\end{enumerate}

}

\frame{

\frametitle{Strengths and ``Wish For''s}

\begin{columns}[T] % align columns
\begin{column}{.4\textwidth}
\begin{block}{Strengths}

\begin{itemize}
\item Classical approach
\item Case Identification
\end{itemize}

\end{block}
\end{column}

\begin{column}{.4\textwidth}
\begin{block}{Wish For}

\begin{itemize}
\item Clarity of family resemblance definitions
\item Completeness of measurement and operationalization
\end{itemize}

\end{block}
\end{column}
\end{columns}

}

\frame{

\frametitle{Concept Definition I}

\begin{itemize}
\item Example of classical definition:\\
``the illegal use of violence or threat by an individual or a group in service of a political agenda with the intention of creating a climate of fear or insecurity''
\item Essential features:
	\begin{itemize}
	\item Illegal use of violence or threat
	\item Carried out by individual or group
	\item Service to a political agenda
	\item Intention to create climate of fear or insecurity
	\end{itemize}
\end{itemize}

}

\frame{

\frametitle{Concept Definition II}

\normalsize

\begin{itemize}\itemsep0.5em
\item Sufficient features?
	\begin{itemize}
	\item Illegal use of (violence or threat)
	\item Carried out by (individual or group)
	\item Service to a political agenda
	\item Intention to create (climate of fear or insecurity)
	\end{itemize}
\item Examples of sufficiency:
	\begin{itemize}
	\item 1,3,4 (allows state actions to be terrorism)
	\item 2,3,4 (allows non-violence to be to be terrorism)
	\item 1,2,4 (allows criminal activity to be terrorism)
	\item 1,2,3 OR 1,2,4 (can be political or fear-inducing)
	\end{itemize}
\end{itemize}

}

% show graphs
% use actual OR and AND notation


\frame{

\frametitle{Operationalization}

\begin{enumerate}
\item Measure features
	\begin{itemize}
	\item Level of measurement
	\item How to score each case on each feature
	\item Be concrete (e.g., carbombing versus railway bombing)
	\end{itemize}
\item Aggregate feature measurements
	\begin{itemize}
	\item Sum? Average? AND logical?
	\item Level of measurement of final scale
	\item Range of possible values
	\item Justify against criticisms/alternatives
	\end{itemize}
\end{enumerate}

}

% measuring violence might require an OR logical operationalization (listing of possible forms of violence)
% might be measures of victims (number hurt or killed); this could then be interval, or reduced categorical, or binary
% might be property damage and/or victims
% attributes might need to be weighted


\frame{}

\section{Case Studies}
\frame{\tableofcontents[currentsection]}

\frame{

\frametitle{Overview}

\begin{itemize}\itemsep0.5em
\item Consistently the most dominant method of social research
\item Often poorly executed (and poorly understood)
\item Three weeks on this topic
	\begin{itemize}
	\item Logic and case selection
	\item Case comparisons
	\item Process-tracing methods
	\end{itemize}
\end{itemize}

}

\frame{

\frametitle{What is a case study?}

\begin{itemize}\itemsep0.1em
\item Definition: ``an intensive study of a single unit for the purpose of understanding a larger class of (similar) units'' (Gerring 2004, 342)
\item Broad uses:
	\begin{itemize}
	\item Description
	\item Induction/Theory development
	\item Theory testing
	\item Exploration of mechanisms
	\item Concept definition and measurement
	\end{itemize}
\end{itemize}

}

% not simply attempting to score a case on a variable but attempting to understand it at a very deep level



% activity???

\frame{

\frametitle{What counts as a case?}

}


\frame{
\frametitle{1: Description}
\begin{itemize}
\item 
\end{itemize}
}

\frame{
\frametitle{2: Theory development}
\begin{itemize}
\item 
\end{itemize}
}

\frame{
\frametitle{3: Theory testing}
\begin{itemize}\itemsep0.5em
\item ``Actual case'' comparisons
\item Fearon's ``Counterfactual method''
\item Process tracing
\end{itemize}
}

\frame{
\frametitle{4: Mechanisms}
\begin{itemize}
\item 
\end{itemize}
}

\frame{
\frametitle{5: Concept Definition}
\begin{itemize}
\item 
\end{itemize}
}






\frame{

\frametitle{Data Collection}

\begin{itemize}
\item Qualitative analysis
	\begin{itemize}
	\item Direction observation
	\item Focus groups
	\item Interviews
	\item Archival/documentary analysis
	\end{itemize}
\item Quantitative analysis
	\begin{itemize}
	\item Surveys
	\item Experiments
	\item Statistical methods
	\item Data mining (e.g., ``big data'')
	\item Data coding
	\end{itemize}
\end{itemize}

}

% describing one or more cases for the purposes of testing a causal hypothesis

% common cases: countries, subnational units, persons, political parties, events (e.g., acts of terrorism), revolutions, transitions to democracy, wars, policies, legislatures, elections
% elections as a cases?




% an instance or observation can be a case of many different things
% September 11th might be a case of terrorism, of structural deficiencies in buildings, of airplane hijacking, of airport security, of intelligence failure, of presidential leadership, of the onset of war, of "rallying around the flag", etc.
% to study something as a case, you need to know what it is a case *of*




\appendix
\frame{}

\end{document}
