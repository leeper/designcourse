\documentclass[14pt]{beamer} %Makes presentation
%\documentclass[handout]{beamer} %Makes Handouts
\usetheme{Singapore} %Gray with fade at top
\useoutertheme[subsection=false]{miniframes} %Supppress subsection in header
\useinnertheme{rectangles} %Itemize/Enumerate boxes
\usecolortheme{seagull} %Color theme
\usecolortheme{rose} %Inner color theme

\definecolor{light-gray}{gray}{0.75}
\definecolor{dark-gray}{gray}{0.55}
\setbeamercolor{item}{fg=light-gray}
\setbeamercolor{enumerate item}{fg=dark-gray}

\setbeamertemplate{navigation symbols}{}
%\setbeamertemplate{mini frames}[default]
\setbeamercovered{dynamics}
\setbeamerfont*{title}{size=\Large,series=\bfseries}

%\setbeameroption{notes on second screen} %Dual-Screen Notes
%\setbeameroption{show only notes} %Notes Output

\setbeamertemplate{frametitle}{\vspace{.5em}\bfseries\insertframetitle}
\newcommand{\heading}[1]{\noindent \textbf{#1}\\ \vspace{1em}}

\usepackage{bbding,color,multirow,times,ccaption,tabularx,graphicx,verbatim,booktabs,fixltx2e}
\usepackage{colortbl} %Table overlays
\usepackage[english]{babel}
\usepackage[latin1]{inputenc}
\usepackage[T1]{fontenc}
\usepackage{lmodern}

%\author[]{Thomas J. Leeper}
\institute[]{
  \inst{}%
  Department of Government\\London School of Economics and Political Science
}

\usepackage{tikz}
\usetikzlibrary{shapes,arrows}

\title{Case Studies}

% Case studies are in-depth examinations of a single manifestation of a political phenomenon and are one of the most common methods of inquiry in political science. What can we do with case studies? How do they help us to understand politics?


\date[]{}

\begin{document}

\frame{\titlepage}

\frame{\tableofcontents}


\section[PS1]{Problem Set 1}
\frame{\tableofcontents[currentsection]}


\frame{

\frametitle{Small Points}

\begin{enumerate}\itemsep0.5em
\item Follow the formatting guidelines
\item Do not copy the questions into the assignment
\item Be succinct but expressive and clear
\item Be anonymous (filename and contents)
\end{enumerate}

}

\frame{

\frametitle{Strengths and ``Wish For''s}

\begin{columns}[T] % align columns
\begin{column}{.4\textwidth}
\begin{block}{Strengths}

\begin{itemize}
\item Classical approach
\item Case Identification
\end{itemize}

\end{block}
\end{column}

\begin{column}{.4\textwidth}
\begin{block}{Wish For}

\begin{itemize}
\item Clarity of family resemblance definitions
\item Completeness of measurement and operationalization
\end{itemize}

\end{block}
\end{column}
\end{columns}

}






\section[Hypothesis Testing]{Conclude Hypothesis Testing}
\frame{\tableofcontents[currentsection]}



\frame{

\frametitle{Fearon's Counterfactuals}

\begin{itemize}\itemsep0.5em
\item Sometimes we cannot test our hypothesis with actual observations % reasons for this: confounding, no counterfactual case, multiple causality
\item What does Fearon suggest we do?
\end{itemize}


}

% thinking counterfactually
	% can we observe the counterfactual (in another case, in the same case at another time, in a collection of other cases)?
	% if not, can we think about what that counterfactual might look like? 
	% see Fearon's use of ``counterfactual method''; thought experiments; hypothetical reasoning



\frame{

\frametitle{A Good Test}

\small

\begin{itemize}\itemsep-0.1em
\item Correct level of analysis
\item Within scope conditions of theory
\item Well-defined concepts
\item Measures of high construct validity, accuracy, and precision
\item Possible to observe any correlation between potential cause and outcome
\item Consistent with or an improvement upon past methods
\item Test using different data than data used to generate theory % distinguish between training set and test set
\end{itemize}

}


\frame{

\frametitle{Some Testing Challenges}

\begin{enumerate}\itemsep0.5em
\item Deterministic and probabilistic causality
\item Effect heterogeneity
\item Multiple causation
\item Equifinality
\item Confirmation or disconfirmation bias
\end{enumerate}

}


\section{Case Studies}
\frame{\tableofcontents[currentsection]}

\frame{

\frametitle{Overview}

\begin{itemize}\itemsep0.5em
\item Consistently the most dominant method of social research
\item Often poorly executed
\item Three weeks on this topic
	\begin{itemize}
	\item Logic and case selection
	\item Case comparisons
	\item Process-tracing methods
	\end{itemize}
\end{itemize}

}



% what is a case study?

% what is it for? description, inductive theory development, testing causal hypotheses (esp. case comparisons, process-tracing, Fearon's ``counterfactual method'')

% most likely (hypothesis confirming)
% least likely (hypothesis weakening)
% deviant cases (hypothesis generating)
% new case, what is something, what explains it?

% make a distinction about a specific instance of a causal relationship and a general pattern of a causal relationship
% tests might only be about a single case but actually apply more generally



\appendix
\frame{}

\end{document}
