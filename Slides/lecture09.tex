\documentclass[17pt]{beamer} %Makes presentation
%\documentclass[handout, 17pt]{beamer} %Makes Handouts
\documentclass[14pt]{beamer} %Makes presentation
%\documentclass[handout]{beamer} %Makes Handouts
\usetheme{Singapore} %Gray with fade at top
\useoutertheme[subsection=false]{miniframes} %Supppress subsection in header
\useinnertheme{rectangles} %Itemize/Enumerate boxes
\usecolortheme{seagull} %Color theme
\usecolortheme{rose} %Inner color theme

\definecolor{light-gray}{gray}{0.75}
\definecolor{dark-gray}{gray}{0.55}
\setbeamercolor{item}{fg=light-gray}
\setbeamercolor{enumerate item}{fg=dark-gray}

\setbeamertemplate{navigation symbols}{}
%\setbeamertemplate{mini frames}[default]
\setbeamercovered{dynamics}
\setbeamerfont*{title}{size=\Large,series=\bfseries}

%\setbeameroption{notes on second screen} %Dual-Screen Notes
%\setbeameroption{show only notes} %Notes Output

\setbeamertemplate{frametitle}{\vspace{.5em}\bfseries\insertframetitle}
\newcommand{\heading}[1]{\noindent \textbf{#1}\\ \vspace{1em}}

\usepackage{bbding,color,multirow,times,ccaption,tabularx,graphicx,verbatim,booktabs,fixltx2e}
\usepackage{colortbl} %Table overlays
\usepackage[english]{babel}
\usepackage[latin1]{inputenc}
\usepackage[T1]{fontenc}
\usepackage{lmodern}

%\author[]{Thomas J. Leeper}
\institute[]{
  \inst{}%
  Department of Government\\London School of Economics and Political Science
}

\usepackage{tikz}
\usetikzlibrary{shapes,arrows,decorations.pathreplacing,calc}

\title{Ethics and Research Integrity}

\date[]{}

\begin{document}

\frame{\titlepage}

\frame{\tableofcontents}

\section{Ethics}
\frame{\tableofcontents[currentsection]}

\frame{
\frametitle{History: Key Moments}

\small

\begin{enumerate}
\item Nuremberg Code (1947)
\item Helsinki Declaration (1964)
\item US 45 CFR 46 (1974) \& ``Common Rule''
	\begin{itemize}\footnotesize
	\item Tuskegee Study (1932-1972)
	\item The Belmont Report (1979)
	\end{itemize}
\item EU Data Protection Directive (1995)
	\begin{itemize}\footnotesize
	\item UK Data Protection Act (1998)
	\item General Data Protection Regulation (2016)
	\end{itemize}
\end{enumerate}
}


\frame{
	\frametitle{Helsinki Declaration}
	
	\small
	\begin{itemize}
	\item Adopted by the World Medical Association in 1964\footnote{\url{http://www.bmj.com/content/2/5402/177}}
	\item Narrowly focused on medical research
	\item Expanded the Nuremberg Code
		\begin{itemize}
		\item Relaxed consent requirements
		\item Risks should not exceed benefits
		\item Institutionalization of ethics oversight
		\end{itemize}
	\item<2-> Do these rules apply to non-experimental research? To non-medical research?
	\end{itemize}

}

\frame{
	\frametitle{Social Science Examples}
	\begin{enumerate}\itemsep2em
	\item \href{https://www.youtube.com/watch?v=yr5cjyokVUs}{Milgram Obedience Study (1961)}
	\item \href{https://www.youtube.com/watch?v=760lwYmpXbc}{Stanford Prison Study (1971)}
	\end{enumerate}
}


\frame{
	\frametitle{The Belmont Report}
	
	\small
	\begin{itemize}
	\item Commissioned by the U.S. Gov't in 1979\footnote{\url{http://www.hhs.gov/ohrp/humansubjects/guidance/belmont.html}}
	\item Three overarching principles:
		\begin{enumerate}
		\item Respect for persons
		\item Beneficence
		\item Justice
		\end{enumerate}
	\item Three policy implications:
		\begin{itemize}
		\item Informed consent
		\item Assessment of risks/benefits
		\item Care for vulnerable populations
		\end{itemize}
	\end{itemize}

}


\frame{
	\frametitle{Informed Consent}
	\begin{itemize}\itemsep0.5em
	\item Persons must consent to being a research subject
	\item<2-> What does this mean in practice?
		\begin{itemize}
		\item What is research?
		\item What is consent?
		\item What is ``informed'' consent?
		\end{itemize}
	\item<3-> Cross-national variations
		\begin{itemize}
		\item Consent forms required in U.S.
		\item Not (legally) required in UK
		\end{itemize}
	\end{itemize}

}

\frame{
\frametitle{Benefits and Harm}
	\begin{itemize}\itemsep2em
	\item What is a ``benefit''?
	\item What is a ``harm''?
	\item How do we balance the two?
	\end{itemize}
}


\frame{
	\frametitle{UK Privacy Law/Ethics}
	\begin{itemize}\itemsep0.5em
	\item Heavily informed by EU law
		\begin{itemize}
		\item EU Data Protection Directive (1995)
		\item UK Data Protection Act (1998)
		\item General Data Protection Regulation (2016)
		\end{itemize}
	\item Data can be processed when:
		\begin{itemize}
		\item Consent is given
		\item Data are used for a ``legitimate'' purpose
		\item Anonymous or confidential
		\end{itemize}
	\item Data generally cannot leave the EU
	\end{itemize}
}


\frame{
	\frametitle{{\normalsize Lots of Other Ethical Questions}}
	\begin{enumerate}
	\item<2-> Funding
	\item<3-> Independence and Politicization
	\item<4-> Vulnerable populations (e.g. children, sick, prisoners, pregnant women, fetuses, refugees)
	\item<5-> Cross-national research
	\item<6-> Participant-observation disclosures
	\item<7-> End uses/users of research
	\item<8-> Others?
	\end{enumerate}
}


\frame{

\Large 

\begin{center}
Questions?
\end{center}

}


\frame{

\frametitle{Activity!}

\begin{itemize}\itemsep1em
\item Read each ethical scenarios
\item Decide what ethical issues are raised by the scenario (if any)
\item Decide what modifications are necessary for the project to be ethically acceptable
\end{itemize}

}




\frame{}

\section{Ethics at LSE}
\frame{\tableofcontents[currentsection]}


\frame{
\frametitle{Research Ethics at LSE}

\small

\begin{itemize}\itemsep0.5em
\item Ethics Code \footnote{\url{https://info.lse.ac.uk/Staff/Divisions/Secretarys-Division/Ethics}}
\item Research Ethics Policy \footnote{\url{https://info.lse.ac.uk/staff/divisions/Secretarys-Division/Ethics/Research-ethics}}
\item Levels of review:
	\begin{enumerate}
	\item Staff: Self-certification
	\item Students: Supervisor certification
	\item LSE Research Ethics Committee
	\item External review
	\end{enumerate}
\end{itemize}

}


\frame{

\begin{center}
\href{http://www.lse.ac.uk/intranet/researchAndDevelopment/researchDivision/policyAndEthics/ethics-annexC.pdf}{\includegraphics[width=\textwidth]{images/lse_ethics_flowchart}}
\end{center}

}

\frame{
\frametitle{Activity!}

\begin{itemize}
\item Complete an LSE Ethics form for your proposed research project
\end{itemize}

}


\appendix
\frame{}

\end{document}
