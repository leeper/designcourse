\documentclass[14pt]{beamer} %Makes presentation
%\documentclass[handout]{beamer} %Makes Handouts
\usetheme{Singapore} %Gray with fade at top
\useoutertheme[subsection=false]{miniframes} %Supppress subsection in header
\useinnertheme{rectangles} %Itemize/Enumerate boxes
\usecolortheme{seagull} %Color theme
\usecolortheme{rose} %Inner color theme

\definecolor{light-gray}{gray}{0.75}
\definecolor{dark-gray}{gray}{0.55}
\setbeamercolor{item}{fg=light-gray}
\setbeamercolor{enumerate item}{fg=dark-gray}

\setbeamertemplate{navigation symbols}{}
%\setbeamertemplate{mini frames}[default]
\setbeamercovered{dynamics}
\setbeamerfont*{title}{size=\Large,series=\bfseries}

%\setbeameroption{notes on second screen} %Dual-Screen Notes
%\setbeameroption{show only notes} %Notes Output

\setbeamertemplate{frametitle}{\vspace{.5em}\bfseries\insertframetitle}
\newcommand{\heading}[1]{\noindent \textbf{#1}\\ \vspace{1em}}

\usepackage{bbding,color,multirow,times,ccaption,tabularx,graphicx,verbatim,booktabs,fixltx2e}
\usepackage{colortbl} %Table overlays
\usepackage[english]{babel}
\usepackage[latin1]{inputenc}
\usepackage[T1]{fontenc}
\usepackage{lmodern}

%\author[]{Thomas J. Leeper}
\institute[]{
  \inst{}%
  Department of Government\\London School of Economics and Political Science
}

\usepackage{tikz}
\usetikzlibrary{shapes,arrows}

\title{Causal Mechanisms and Process-Tracing}

% Aside from knowing that one thing (X) caused another thing (Y), we often want to know how that causal process worked. This is the study of ``causal mechanisms''. How do we study causal mechanisms to gain a deeper understanding of causal relationships in politics? How do we study the process by which a causal effect plays out?


\date[]{}

\begin{document}

\frame{\titlepage}

\frame{\tableofcontents}


\section{Review}
\frame{\tableofcontents[currentsection]}

\frame{
	\frametitle{Review Case Studies}
	\begin{itemize}
	\item Many uses of case studies
	\item In case comparisons (last week), we focused on scoring cases on variables to test theories \textit{between cases}
	\item Now we focus on \textit{within-case} comparisons
	\end{itemize}
}


\frame{
\frametitle{{\large Causal Process Observations}}

\normalsize

\begin{itemize}\itemsep0.5em
\item Definition: ``An insight or piece of data that provides information about the context, process, or mechanism, and that contributes distinctive leverage in causal inference''\footnote{Brady and Collier 2004, p.277}
\item Might be used to:
	\begin{itemize}
	\item Inductively generate hypotheses about potential causal relationships
	\item Deductively test a chain of causal relationships
	\end{itemize}
\end{itemize}
}


\frame{}

\section{Mechanisms}
\frame{\tableofcontents[currentsection]}


\frame{
\frametitle{Four (or five) principles of causality\footnote{From Kellstedt and Whitten}}
\begin{enumerate}
\item Correlation
\item Nonconfounding
\item Direction (``temporal precedence'')
\item \textbf<2->{Mechanism}
\item (Appropriate level of analysis)
\end{enumerate}
}

% that is to say: how are cause and outcome linked causally?




% causal graphs

% causality is complicated
% do we care about mechanisms? how deep do we want to go into a mechanism?


% mechanisms allow us to talk about direct and ``mediated'' effects


\frame{}

\section{Process Tracing}
\frame{\tableofcontents[currentsection]}

% at its most basic level, it answers ``what happened?'' (i.e., it is descriptive history)
% the difference, however, is that it is about causal inference, which is implicitly about within-case counterfactuals
% generally, these are treated with a deterministic perspective on causality (if this hadn't happened, what would have happened instead?)
% class examples: Sherlock Holmes stories are process-tracing tests of how murders (or other crimes) that link a potential cause (a murderer) to an outcome (a death) via a series of causal steps that leave behind pieces of evidence

% application of logic and ``counterfactual'' method
% update beliefs about counterfactuals in sequence

% inductive versus deductive approach
% process tracing as a standalone method versus as a supplement to other methods
	% for exmaple, often use very aggregated data to establish a relationship and process tracing to document how it comes about


% high amounts of uncertainty

\section{Preview}
\frame{\tableofcontents[currentsection]}


\frame{
\frametitle{Research Design Proposal}

\begin{itemize}\itemsep0.5em
\item Instructions posted on Moodle
\item Use class sessions to discuss topics
\item Don't worry about design now
\item Focus instead on topics, questions, and theories
\end{itemize}
}


\frame{
\frametitle{Coming weeks (MT and LT)}

\begin{itemize}\itemsep0.5em
\item Methods of data collection
	\begin{enumerate}
	\item Text
	\item Interviews/Surveys
	\item Observation
	\end{enumerate}
\item Problem Set 4 (due in December)
\item Shift to methods of quantitative data analysis
\end{itemize}
}


\appendix
\frame{}

\end{document}
