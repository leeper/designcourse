\documentclass[14pt]{beamer} %Makes presentation
%\documentclass[handout]{beamer} %Makes Handouts
\usetheme{Singapore} %Gray with fade at top
\useoutertheme[subsection=false]{miniframes} %Supppress subsection in header
\useinnertheme{rectangles} %Itemize/Enumerate boxes
\usecolortheme{seagull} %Color theme
\usecolortheme{rose} %Inner color theme

\definecolor{light-gray}{gray}{0.75}
\definecolor{dark-gray}{gray}{0.55}
\setbeamercolor{item}{fg=light-gray}
\setbeamercolor{enumerate item}{fg=dark-gray}

\setbeamertemplate{navigation symbols}{}
%\setbeamertemplate{mini frames}[default]
\setbeamercovered{dynamics}
\setbeamerfont*{title}{size=\Large,series=\bfseries}

%\setbeameroption{notes on second screen} %Dual-Screen Notes
%\setbeameroption{show only notes} %Notes Output

\setbeamertemplate{frametitle}{\vspace{.5em}\bfseries\insertframetitle}
\newcommand{\heading}[1]{\noindent \textbf{#1}\\ \vspace{1em}}

\usepackage{bbding,color,multirow,times,ccaption,tabularx,graphicx,verbatim,booktabs,fixltx2e}
\usepackage{colortbl} %Table overlays
\usepackage[english]{babel}
\usepackage[latin1]{inputenc}
\usepackage[T1]{fontenc}
\usepackage{lmodern}

%\author[]{Thomas J. Leeper}
\institute[]{
  \inst{}%
  Department of Government\\London School of Economics and Political Science
}

\usepackage{tikz}
\usetikzlibrary{shapes,arrows}

\title{Measurement: Concepts in Practice}

% To study something, we need to be able to observe and measure it. How do we \textit{operationalize} concepts so that we can study political phenomena? What are challenges of measuring concepts? How do we assign quantitative values to observations?

\date[]{}

\begin{document}

\frame{\titlepage}

\frame{\tableofcontents}


\section{Review}
\frame{\tableofcontents[currentsection]}


\frame{

\frametitle{Concept Definition}

\begin{itemize}
\item Classical approach (minimal, maximal, ordinal)
\item Family resemblance approach
\end{itemize}

}

% review family resemblence, esp. joint sufficiency and necessity


\frame{
\frametitle{Gerring's Criteria}

\begin{enumerate}
\item Resonance (face validity)
\item Domain/scope % ladder of generality
\item Consistency
\item Fecundity
\item Differentiation
\item Causal utility
\item Operationalization
\end{enumerate}
}

% divide these up and have someone present each one


% Resonance: Is the concept intuitive? Avoid neologism (Gerring p.118); we don't need new terms for the sake of having new terms

% Domain: The contexts in which this concept resonates and applies. Example: "vouchers". What does this mean? This has a clear meaning in the United States, in debates about education policy, but that is a narrow domain. Contrast this with democracy or terrorism, which are concepts with broad domains.

% Consistency: We can change the definition of a concept by adding, removing, or modifying attributes; Contrast with "slippage" or "stretching"

% Fecundity: Fertility or fruitfulness; Neologisms are prone to lacking fecundity

% Differentiation: How does this concept contrast or help create distinctions between existing sets of phenomena? Example: *Opinion* is a summary evaluation of a particular object; *Value* is a belief about a desired end-state of the world

% Causal Utility: Concept has to be useful for making a causal argument; this may require a concept to be more specific or have higher intensity than we would prefer because we need to *use* the concept

% Operationalization: concepts have to be measurable; we'll talk about this next week





\section{Measurement}
\frame{\tableofcontents[currentsection]}


\frame{
\frametitle{An Example: Opinion}

An example

Definition: *Opinion* is a summary evaluation of a particular object

Only one feature: evaluation or favorability

Operationalization?


Agree/disagree

Oppose/support

Degree of favorability

Warm/cool

Positive/negative

Implicit/explicit

How many scale points?



}





\frame{

Recall definition of \textit{variable}:

\begin{itemize}\itemsep0.5em
\item \textit{Observation}: A case or unit (e.g., person, country)
\item \textit{Variable}: A dimension that describes an observation (e.g., income)
\end{itemize}

}

Operationalization = Creating measures for concepts






Measures follow from concept definitions

Definitions of democracy from Gerring: How do we operationalize these things?

do you add? do you average? do you count some indicators more than others? are some necessary (as in Gerring's ordinal definition)





Definition -> Feature -> Measure(s)/indicator(s)

Concept: Democracy
Feature/Dimension: Free and fair elections
Indicator: Validation of an election by international election observers


can be one or more measures or indicators
these may or may not "scale"






Activity around this



Conceptual Clarity 

Content validity: is this a measure of the thing you want to measure? Or is it a measure of something else?

WRONG DEFINITION: Construct Validity (Kellstedt and Whitten, p.102): degree to which a measure is related to other measures where a relationship is expected by theory

Their ``content validity'' is SSC's ``construct validity''

Convergent or discriminant validity: two (or more) measures of the same concept are highly correlated and two (or more) measures of distinct concepts are not correlated
 - Measures of distinct concepts may be correlated if they are causally related to one another, so simple correlations do not mean two measures are necessarily of the same concept


Threats to Construct Validity (Shadish, Cook, and Campbell Table 3.1 (p.73))

- Bad concept definition

- Mono-operation and mono-method bias: when the operationalization mismeasures the concept or when the operationalization itself becomes part of the concept (e.g., self-reported income versus actual income)

- Experimenter expectancies: does the implementation of the experiment actually expose units to things other than the concept of interest (e.g., by encouraging particular types of behaviour aside from the behaviour of interest)

- Compensatory equalization and rivalry: Knowledge of treatment status actually changes behaviour

- Treatment diffusion: Stable unit treatment value assumption from Holland



Quantitative measures versus Qualitative measures

Example: democracy

categorical

ordinal

numeric/interval



Accuracy (unbiased; true)
% true kilogram; ``ideal type'' (Public Domain, https://commons.wikimedia.org/wiki/File:MassStandards_005.jpg)

Precise (low variance; highly certain)

Reliable (dependable; replicable; repeatable)
% multiple thermometers all converge on the same answer; intercoder reliability

% target examples
% accuracyprecision.png Wikimedia user Guam, https://commons.wikimedia.org/wiki/File:Accuracy_and_precision_example.jpg


Multiple measures


Important to remember that theory always plays out at the level of concepts
Analysis plays out at the level of measures
So, we are always looking for the observable implications of theory and the conceptual meaning of observable measures and their theoretical importance



\appendix
\frame{}

\end{document}
