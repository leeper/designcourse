\documentclass[11pt, a4paper]{article}
\usepackage[margin=1in]{geometry}
\usepackage{setspace}
\setlength{\parindent}{0cm}
\setlength{\parskip}{0.2cm}


\begin{document}

\thispagestyle{empty}

\begin{center}
\textbf{UKIP 2015 Manifesto (p.5)}
\end{center}


If only all politicians could believe in Britain as UKIP does. If only they could share our positive vision of Britain as a proud, independent sovereign nation, a country respected on the world stage, a major player in global trade, with influence and authority when it comes to tackling the pressing international issues of the day.

This manifesto is our blueprint for a Britain released from the shackles of the interfering EU. This makes it markedly different to those of the other main parties. While we see a free, prosperous, healthy, international future for Britain, their cowardice binds our country to a failing super-state that tells us what to do and does not listen to what we want.

Our manifesto also throws down the gauntlet to those who have ridiculed us, jeered at us and lied about our voters, our people and our policies. It tells the truth about what UKIP stands for.

It takes care of Britain’s finances too. Taxpayers could get so much better value for their money if we left the EU, made reasonable cuts to the foreign aid budget, replaced the unfair Barnett formula, scrapped HS2, ended `health tourism' and cut the cost of government. These are our plans and they give us a budget to invest in Britain the other parties can only dream of. Without adding a penny to our burgeoning national debt, without cutting vital services and without raising taxes, we can, over the course of the next parliament:

\begin{itemize}

\item Invest an extra \pounds 12 billion into the NHS; put \pounds 5.2 billion more into social care; build a dedicated military hospital and abolish hospital parking charges
\item End income tax on the minimum wage; cut income taxes for middle earners; scrap inheritance tax; abolish the ‘bedroom tax’ and increase the transferable personal tax allowance for married couples and civil partners
\item Fund 6,000 additional posts spread between the police service, the prison service and the Border Agency
\item Cut business rates for small businesses
\item Waive tuition fees for students taking a degree in science; technology; engineering; maths or medicine
\item Increase defence spending to 2 per cent of GDP to honour our NATO obligations
\item Invest \pounds 1.5 billion into mental health and dementia services
\item Pay carers an extra \pounds 572 a year
\item Build 500 affordable rent homes every year and eight halfway house hostels for homeless veterans
\item Remove stamp duty on the first \pounds 250,000 for new homes built on brownfield sites.

\end{itemize}

Our fiscal plans identifying how we will fund these policies have been rigorously and independently assessed by the Centre for Economic and Business Research (Cebr). The Cebr has been extremely diligent and has challenged us frequently, so voters can be assured our manifesto commitments are sound and affordable.

Ours is an amazing country, but it could be better still. As Mark Reckless said when he joined UKIP last year: ``we are more than just a star on someone else’s flag.''

Britain is great: if you believe this too, then believe in us and vote UKIP.

\end{document}
